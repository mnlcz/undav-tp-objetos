\section{Funcionamiento del sistema}\label{sec:funcionamiento-del-sistema}
En el siguiente apartado vamos a explicar cómo se comportaría nuestro programa en una posible ejecución real.
Dado que planteamos una ejecución por consola, vamos a basar el siguiente apartado en la funcionalidad prevista en el
main (\lstinline|Main.java|), la cual se basa en el uso de la clase \lstinline|App| como entidad principal.

El recorrido por las distintas funcionalidades se va a realizar mediante opciones, un comportamiento esperado dado que
estamos trabajando en consola.

\subsection{Inicio}\label{subsec:inicio}
Al inicio el programa le brinda al usuario (en esta etapa refiriéndose a usuario/consumidor de la aplicación, no a un
usuario registrado en el sistema) una serie de opciones, siendo las más importantes:
\begin{itemize}
    \item Inicio de sesión.
    \item Registro de un nuevo usuario.
\end{itemize}
Algo \textbf{fundamental} a tener en cuenta con respecto al resto de opciones:
\begin{itemize}
    \item Agregar contenido a la plataforma.
    \item Quitar contenido de la plataforma.
\end{itemize}
Es el hecho de que nosotros planteamos que los administradores del sistema sean quienes agreguen los contenidos y que los
usuarios simplemente hagan uso de ellos, ya sea puntuándolos o agregándolos a su perfil.
Tal y como funciona una plataforma como Netflix por ejemplo, los administradores agregan las películas y los usuarios las
consumen.
La existencia de estas opciones en nuestra ejecución se da básicamente por temas de testing/conveniencia, ya que, como se
dijo anteriormente, el usuario no debería crear o quitar contenidos, pero nos es imposible simular dicho comportamiento.

\subsection{Usuario iniciado}\label{subsec:usuario-iniciado}
Cuando el sistema detecta que el usuario ha ingresado, las opciones brindadas pasan a ser otras.
Puntualmente:
\begin{itemize}
    \item Gestionar contenidos: agregar contenidos vistos o a la wishlist, puntuar, etc.
    \item Gestionar amigos: agregar o quitar amigos.
\end{itemize}
Creemos que la lógica en el main está lo suficientemente clara como para abstenernos de hacer una explicación paso por
paso en este informe.
Dado que la ejecución sigue fluyendo en distintos prompts de diversas opciones, recomendamos consultar \lstinline|Main.java|
para una vista clara de todo.
Concluimos detallando que este flujo de ejecución continua hasta que el usuario cierre sesión/cierre el programa.
